\documentclass[11pt, a4paper, twoside]{ctexbook}
\usepackage{amsmath, amsthm, amssymb, bm, graphicx, hyperref, mathrsfs}
\usepackage{geometry}
\usepackage{ulem}

\usepackage{listings}
\usepackage{tcolorbox}
\usepackage{colortbl}
\usepackage{geometry}
\geometry{margin = 1in}

\tcbuselibrary{most}

\usepackage{bm}
\usepackage{CJKulem}
\usepackage{color} 
\usepackage {xcolor}

\newtcbtheorem[number within=section]{mytheorem}{定理}
{enhanced, breakable,colframe=black!75!white ,fonttitle=\bfseries}{th}
\newtcbtheorem[number within=section]{mycorollary}{推论}
{enhanced, breakable,colframe=black!75!white ,fonttitle=\bfseries}{th}
\newtcbtheorem[number within=section]{mydefinition}{定义}
{enhanced, breakable,colframe=black!60!white ,fonttitle=\bfseries}{th}
\newtcbtheorem[number within=section]{myproposition}{命题}
{enhanced, breakable,colframe=black!45!white ,fonttitle=\bfseries}{th}
\usepackage{mathrsfs}
\usepackage{tipa}
\usepackage{amssymb}
\usepackage{geometry}
\usepackage{hyperref}
\usepackage{ctex}
\usepackage{amsmath}
\usepackage{amsthm}

\usepackage{tikz-cd}


\usepackage{titlesec}
\usepackage{titleps}
\usepackage{tikz}

\newcommand{\chapnumfont}{%
	\fontsize{100}{100}\usefont{T1}{ptm}{b}{n}%
}

\colorlet{chapbgcolor}{gray!75}
\colorlet{chapnumcolor}{black!60}

\newcommand{\chaptitlenumbered}[1]{%
	\begin{tikzpicture}
		\fill[chapbgcolor!70,rounded corners=0pt] (0,2.3) rectangle (\linewidth,0);
		\node[
		align=right,
		anchor=south east,
		inner sep=8pt,
		font=\huge\normalfont\bfseries
		] at (0.987\linewidth,0) {\strut#1};
		\node[
		align=right,
		font=\fontsize{60}{62}\usefont{OT1}{ptm}{b}{it},
		text=chapnumcolor
		] at (0.975\linewidth,2.1) {\thechapter};
	\end{tikzpicture}%
}

\newcommand{\chaptitleunnumbered}[1]{%
	\begin{tikzpicture}
		\fill[chapbgcolor!70,rounded corners=0pt] (0,2.3) rectangle (\linewidth,0);
		\node[
		align=right,
		anchor=south east,
		inner sep=8pt,
		font=\huge\normalfont\bfseries
		] at (0.987\linewidth,0) {\strut#1};
	\end{tikzpicture}%
}

\titleformat{name=\chapter}[display]
{\normalfont\huge\bfseries\sffamily}
{}
{0pt}
{\chaptitlenumbered}
\titleformat{name=\chapter,numberless}[display]
{\normalfont\huge\bfseries\sffamily}
{}
{25pt}
{\chaptitleunnumbered}
\titlespacing*{\chapter}
{0pt}
{-126pt}
{33pt}

\setlength\headheight{15pt}

\geometry{a4paper,scale=0.80}
\title{{\Huge{\heiti{实分析习题选解}}}\\\mathbf{Notes and Exercises}}
\author{被电信诈骗的小董}
\date{\today}
\linespread{1.5}
\newtheorem{theorem}{定理}[section]
\newtheorem{definition}[theorem]{定义}
\newtheorem{lemma}[theorem]{引理}
\newtheorem{corollary}[theorem]{推论}
\newtheorem{example}[theorem]{例}
\newtheorem{proposition}[theorem]{命题}
\newenvironment{proof2}{{\noindent\heiti 证明}}{\hfill $\Box $\par}
\newenvironment{proof3}{{\noindent\heiti 解}}{\hfill $\Box $\par}
\begin{document}
	
	\maketitle
	
	\pagenumbering{roman}
	\setcounter{page}{1}
	
	\begin{center}
		\huge\textbf{前言}
	\end{center}~\
	
	这是笔记的前言部分. 
	~\\
	\begin{flushright}
		\begin{tabular}{c}
			小董\\
			\today
		\end{tabular}
	\end{flushright}
	
	\newpage
	\pagenumbering{Roman}
	\setcounter{page}{1}
	\tableofcontents
	\newpage
	\setcounter{page}{1}
	\pagenumbering{arabic}
	
	\chapter{集合与运算}
	
	
	
	\section{集合及其运算 }
	
	\begin{myproposition}{习题5}{}
		设${A_{n}}$是一集合列,令$$B_{1}=A_{1}, B_{i}=A_{i}\bigg \backslash \bigcup_{j=1}^{i-1} {A_{j}}\quad (i>1)$$
		证明:${B_{n}}$互不相交,且对任意的$n$,有$\bigcup\limits_{i=1}^{n} {A_{i}}=\bigcup\limits_{i=1}^{n} {B_{i}}$.
	\end{myproposition}
	\begin{proof2}
		1°先证明$B_{i}$互不相交,
		
		
		$\forall i\neq j$,不妨设$i>j$,
		$\forall x\in A_{i}$且$x\notin A_{k}(k\leq i-1)$,有$x\notin B_{k}(k\leq i-1)$,即$x\notin B_{j}$,从而$B_{i}\bigcap B_{j}=\emptyset $.
		
		
		2°再证明两个集合相等.
		
		
		$\forall n,\forall x \in \bigcup\limits_{i=1}^{n} {B_{i}},\exists k,x\in B_{k}$,则 $x\in A_{k}\subset \bigcup\limits_{i=1}^{n} {A_{i}} $.
		
		
		
		$\forall n,\forall x \in \bigcup\limits_{i=1}^{n} {A_{i}},\exists k,x\in A_{k}$,
		
		
		
		若$x\notin \bigcup\limits_{i=1}^{k-1}A_{i}$,则$x\in B_{k}\subset \bigcup\limits_{i=1}^{n} {B_{i}}$.
		
		
		
		若$x\in \bigcup\limits_{i=1}^{k-1}A_{i}$,则$\exists k_{1},k_{2},\cdots \mathrm{s.t.}x\in A_{k_{i}}$.
		
		
		
		取$m=min\{ k_{1},k_{2},\cdots \} $,知$x\in A_{m},x\notin A_{m-1},A_{m-2},\cdots,A_{1}$.故$x\in B_{m}\subset \bigcup\limits_{i=1}^{n} {B_{i}}$.
		
		
		
		综合即有$\bigcup\limits_{i=1}^{n} {A_{i}}=\bigcup\limits_{i=1}^{n} {B_{i}}$.证毕.
	\end{proof2}
	
	
	\begin{myproposition}{习题7}{}
		设$f_{n}(x)$是区间$(a,b)$上的单调递增实函数列.证明:
		$$\forall a\in \mathbb{R} ,\quad \left\{x\big|\lim_{n \to \infty}f_{n}(x)>a\right\}=\bigcup\limits_{n=1}^{\infty}\{x\mid f_{n}(x)>a\}$$
	\end{myproposition}
    \begin{proof2}
		一方面,由数列递增性质以及极限的保序性,若$f_{n}(x)>a$,则有:$\displaystyle\lim_{n \to \infty}f_{n}(x) \geq f_{1}(x) >a$.


       另一方面,记$\displaystyle\lim_{n \to \infty}f_{n}(x)=a'$,则$\forall m \in \mathbf{N},\exists N$,当$n\geq N$时,有$\mid f_{n}(x)-a'\mid < \displaystyle\frac{1}{m}$.


        则$f_{n}(x)>a'-\displaystyle\frac{1}{m}$.取$m=\left[\displaystyle\frac{1}{a'-a}\right]+1$,则$f_{n}(x)>a'-\displaystyle\frac{1}{m}\geq a$.


        即两侧集合互相包含.证毕.
	\end{proof2}


	\begin{myproposition}{习题10}{}
		设$f_{n}(x)$是$\mathbf{R}$上的函数列,集合$E \subset \mathbf{R}$,已知:
        $$\lim_{n \to \infty}f_{n}(x)=\mathbf{X_{E}}$$
        记$E_{n}=\{x\mid f_{n}(x)\geq \displaystyle\frac{1}{2}\}$,求集合$\displaystyle\lim_{n \to \infty}E_{n}$.
	\end{myproposition}
    \begin{proof2}
		当$n \to \infty$时,$E_{n}$中元素都是函数值不小于$\displaystyle\frac{1}{2}$的点.猜测$\displaystyle\lim_{n \to \infty}E_{n}=E$. 


        下证这两个集合相等.


        $\forall x \in E,\displaystyle\lim_{n \to \infty}f_{n}(x)=1$,则$\exists N_{1}$,当$n\geq N_{1}$时,有:
        $$\mid f_{n}(x)-1\mid <\displaystyle\frac{1}{2}\Rightarrow f_{n}(x)>\frac{1}{2}\Rightarrow x \in E_{n},n\geq N_{1}\Rightarrow x \in \lim_{n \to \infty}E_{n}$$
        
		
		$\forall x \in \displaystyle\lim_{n \to \infty}E_{n}$,则$\exists N_{2}$,当$n\geq N_{2}$时,有:
        $$x \in E_{n} \Rightarrow f_{n}(x)\geq \frac{1}{2},\forall n \geq N_{2} \Rightarrow \lim_{n \to \infty}f_{n}(x)=1 \Rightarrow x \in E$$
       
		
		故$\displaystyle\lim_{n \to \infty}E_{n}=E$.
	\end{proof2}


	\begin{myproposition}{习题11}{}
		设$A_{n}=\{\displaystyle\frac{m}{n}\mid m\in \mathbf{Z}\},n=1,2,…$,证明:
        $$\varliminf_{n \to \infty}A_{n}=\mathbf{Z};\varlimsup_{n \to \infty}A_{n}=\mathbf{Q}.$$
	\end{myproposition}
    \begin{proof2}
		1°一方面,$\forall x \in \displaystyle\varliminf_{n \to \infty}A_{n},x \in \displaystyle\bigcup_{n=1}^{\infty}\bigcap_{k=n}^{\infty}A_{k}.\exists t,x\in \displaystyle\bigcap_{k=t}^{\infty}A_{k}.$即$x\in A_{t},A_{t+1},…$


		设$x=\displaystyle\frac{a_{1}}{t}=\displaystyle\frac{a_{2}}{t+1}(a_{1},a_{2}\in \mathbf{Z})$,变形得$(a_{2}-a_{1})t=a_{1}$,易见$a_{1}\equiv 0(mod\quad0)$.则$x\in \mathbf{Z}$.


		另一方面,$\forall x \in \mathbf{Z}$,易见$x$位于$A_{k}$中的第$kx$位,故$x\in A_{k},\forall k \in \mathbf{N}$.则$x\in \displaystyle\varliminf_{n \to \infty}A_{n}.$


		于是我们证明了$\displaystyle\varliminf_{n \to \infty}A_{n}=\mathbf{Z}$.


		2°一方面,$\forall x \in \displaystyle\varlimsup_{n \to \infty}A_{n},x \in \displaystyle\bigcap_{n=1}^{\infty}\bigcup_{k=n}^{\infty}A_{k}.\exists t,x\in \displaystyle\bigcup_{k=t}^{\infty}A_{k}.$则$x \in \displaystyle\bigcup_{k=1}^{\infty}A_{k}.$而$A_{k}\subset \mathbf{Q}$,故$x\in \mathbf{Q}$.


		另一方面,$\forall x \in \mathbf{Q}$,记$x=\displaystyle\frac{p}{q},(p,q)=1$.由$x=\displaystyle\frac{p}{q}=\displaystyle\frac{np}{nq}$知$x\in A_{nq}$,令$n \to \infty$,有$x \in \displaystyle\lim_{n\to \infty}\displaystyle\bigcup_{k=n}^{\infty}A_{k}$,即$x\in \displaystyle\varlimsup_{n \to \infty}A_{n}.$
		
		
		于是我们证明了$\displaystyle\varlimsup_{n \to \infty}A_{n}=\mathbf{Q}$.证毕.		
	\end{proof2}


	\begin{myproposition}{习题12}{}
		设$0<a_{n}<1<b_{n}.n=1,2,…$,已知${a_{n}},{b_{n}}$单调下降,且$\displaystyle\lim_{n \to \infty}a_{n}=0,\displaystyle\lim_{n \to \infty}b_{n}=1$.
		证明:$$\lim_{n\to \infty}[a_{n},b_{n}]=(0,1].$$
	\end{myproposition}
    \begin{proof2}
		往证:$\displaystyle\varlimsup_{n\to \infty}[a_{n},b_{n}]\subset(0,1]\subset\displaystyle\varliminf_{n\to \infty}[a_{n},b_{n}]$.即证:$\displaystyle\bigcap_{n=1}^{\infty}\bigcup_{k=n}^{\infty}[a_{k},b_{k}]\subset(0,1]\subset \displaystyle\bigcup_{n=1}^{\infty}\bigcap_{k=n}^{\infty}[a_{k},b_{k}]$.
		
		
		一方面,若$x\in \displaystyle\bigcap_{n=1}^{\infty}\bigcup_{k=n}^{\infty}[a_{k},b_{k}]$,则$x\in \displaystyle\bigcup_{k=n}^{\infty}[a_{k},b_{k}],\forall n \in \mathbf{N}$.
		
		
		1° 若$x\leq 0$,则$x < a_{n},x\notin \displaystyle\bigcup_{k=n}^{\infty}[a_{k},b_{k}]$,矛盾!
		
		
		2° 若$x>1$,由$\displaystyle\lim_{n \to \infty}b_{n}=1$知,$\forall \varepsilon >0,\exists N$,当$n>N$时,$\mid b_{n}-1\mid<\varepsilon$.取$\varepsilon=x-1$,则有$b_{n}<x$,故$x\notin \displaystyle\bigcup_{k=n}^{\infty}[a_{k},b_{k}]$,矛盾!则$x\in(0,1]$.左式得证.
		
		
		另一方面,若 $x\in(0,1]$.$\forall \varepsilon>0,\exists n$,当$k\geq n$时,$\mid a_{k}-0\mid<\varepsilon$,即$a_{n}<\varepsilon$.取$\varepsilon=x$,则$a_{k}<x\leq 1 \leq b_{k} \Rightarrow x \in [a_{k},b_{k}](k\leq n)$.故$x \in \displaystyle\bigcap_{k=n}^{\infty}[a_{k},b_{k}] \Rightarrow x \in \displaystyle\bigcup_{n=1}^{\infty}\bigcap_{k=n}^{\infty}[a_{k},b_{k}]$.右式得证.证毕.
	\end{proof2}


    \section{映射}

	\begin{myproposition}{习题13}{}
	    证明$\mathbf{R^2}$中至少有一个圆周不含有理点.
    \end{myproposition}
	\begin{proof2}
		用反证法。假设每个圆周都含有有理点,则$\forall r>0,\{(x,y)\mid x^2+y^2=r^2\}\bigcap \mathbf{Q}\neq \emptyset.$

        
		故可以找到 $\mathbf{R^{+}} \to \mathbf{Q^2},r \mapsto  (x,y)$的一个单射.则$c \leq {N}_{0}\times {N_{0}}={N_{0}}$.这与$c>N_{0}$矛盾.证毕. 
    \end{proof2}


	\begin{myproposition}{习题15}{}
		给出$\mathbf{R\backslash Q}$到$\mathbf{R}$之间的一一对应.
    \end{myproposition}
	\begin{proof3}
		在大多数情况下,关于不可数$\to$不可数$+$可数的构造,都是绝大部分个元素映射到自身,剩下可列个元素错位映射.


		现给出如下构造:$f_{1}(x):\mathbf{Q}+\sqrt{2} \to  \mathbf{Q}\bigcup \mathbf{Q}+\sqrt{2}$(两个集合均可列)
		$$f(x)=\left\{\begin{matrix}
			x \qquad & x\in \mathbf{R\backslash Q\bigcup \mathbf{Q}+\sqrt{2}}\\
			f_{1}(x) & x\in \mathbf{Q}+\sqrt{2}
		   
		   \end{matrix}\right.$$
		下面是一个$(0,1]\to (0,1)$的构造.
		$$f(x)=\left\{\begin{matrix}
			x&other\\\displaystyle\frac{1}{2}x & \exists n \in \mathbf{N},\displaystyle\frac{1}{x}=2^{n}
		\end{matrix}\right.$$    
	\end{proof3}


	\begin{myproposition}{习题17}{}
		有理系数多项式的实零点称为代数数,不是代数数的实数称为超越数.证明:全体代数数的集合的势为$N_{0}$,而超越数的势为$c$.
    \end{myproposition}
	\begin{proof2}
		1°先证明:全体代数数的集合的势为$N_{0}$.记全体$n$次有理系数多项式的全体为$A_{n},n=1,2,…$.


		则全体有理系数多项式的全体$A=\displaystyle\bigcup_{n=1}^{\infty}A_{n}.$注意到:$$A_{n}=\{f\mid f=a_{0}x^{0}+a_{1}x^{1}+…+a_{n}x^{n},a_{i}\in\mathbf{Z},a_{n}\in \mathbf{N}\}.$$
		
		
		易见$A_{n}$与$(a_{1},a_{2},…,a_{n})$对等,故$A_{n}$的势为$N_{0}^{n}=N_{0} \Rightarrow A_{n}$可列$\Rightarrow \displaystyle\bigcup_{n=1}^{\infty}A_{n}$可列.
		
		
		而$\forall f \in  A_{n},f$至多有$n$个实零点,即$A_{n}$对应至多$n$个代数数,$\displaystyle\bigcup_{n=1}^{\infty}A_{n}\times \{1,2,…,n\}\to $代数数$\mathbf{M}$.
		
		
		故$\mid \mathbf{M}\mid =N_{0}$.
		
		
		2°证明全体超越数构成的集合$\mathbf{G}$的势为$c$是类似第15题的,这是因为$\mathbf{G}=\mathbf{R}\backslash\mathbf{M}\sim\mathbf{R}$.证明从略.
	\end{proof2}


	\begin{myproposition}{习题18}{}
		$\mathbf{R}$上全体开集记为$\mathcal{T} $,证明$\mid \mathcal{T}\mid=c$.    
	\end{myproposition}
	\begin{proof2}
		一方面,$(-1,1)\in \mathcal{T}\Rightarrow \mid \mathcal{T}\mid \geq \mid (-1,1) \mid = c$.
		
		
		另一方面,$\forall \mathbf{A} \in \mathcal{T},f:\mathbf{A} \mapsto \mathbf{A}\bigcap \mathbf{Q},\mathcal{T}\to 2^{\mathbf{Q}}$是单射,故$\mid \mathcal{T}\mid\leq c$.
		
		
		证毕.
		
		
		$Tips:$本题还有一般解法,需要如下两个漂亮的结论.读者可用该结论自行证明本题.
		$$\mid A_{i}\mid =c \Rightarrow \mid \displaystyle\prod _{i=1}^{\infty}A_{i}\mid =c.$$
		
		
		一方面,由$x\mapsto (f_1{x},f_{2}(x),…,f_{n}(x),…),\mathbf{R}\to \displaystyle\prod _{i=1}^{\infty}A_{i}$为单射知$\mid \displaystyle\prod _{i=1}^{\infty}A_{i}\mid \geq c.$
		
		
		另一方面,设$\displaystyle\prod _{i=1}^{\infty}A_{i}=(a^{(1)},a^{(2)},…)$,其中$a^{(i)}=0.a_{1}^{(i)}a_{2}^{(i)}a_{3}^{(i)}……$,
		
		
		显然$A_{i}=(0,1]$,满足$\mid A_{i}\mid =c$.从$a^{(i)}=0.a_{1}^{(i)}a_{2}^{(i)}a_{3}^{(i)}……$中
		
		
		我们选择$a^{(i)}$的小数点后第$i$位作为数$a$的小数点后第$i$位,则有$(a^{(1)},a^{(2)},…)\mapsto a,\displaystyle\prod _{i=1}^{\infty}A_{i}\to \mathbf{R}$
		为单射,故$\mid \displaystyle\prod _{i=1}^{\infty}A_{i}\mid \leq c.$证毕.
		
		
		此结论也即22题.
		
		
		$$\mid A_{i}\mid =c \Rightarrow \mid \displaystyle\bigcup _{i=1}^{n}A_{i}\mid =c.$$
		
		
		显然.将各个集合映射成数轴上互不相交但相连的区间即可,这里不做证明.
	\end{proof2}

	\begin{myproposition}{习题20}{}
		设$\mathbf{X}$是无限集合,给定:$f:\mathbf{X}\to \mathbf{X},f$不是恒同映射.
		证明:$\exists \mathbf{E},\mathbf{E}$是$\mathbf{X}$的非空真子集且$f(\mathbf{E})\subset \mathbf{E}$.    
	\end{myproposition}
	\begin{proof2}
		由于$f:\mathbf{X}\to \mathbf{X},f$不是恒同映射,故$\exists a \in \mathbf{X},f(a)\neq a$但$f(a)\in \mathbf{X}$.


		令$\mathbf{E}=\{f(a),f^{2}(a),…\}$,则$f(\mathbf{E})\subset\mathbf{E}.$


		由归纳原理可知,$\forall n,f^{n}(x)\in \mathbf{X}$,即$\mathbf{E}\subset \mathbf{X}$.


		下面只需证明:$\mathbf{E}\neq \mathbf{X}$.


		考虑反证法.假设$\mathbf{E}= \mathbf{X}$,则$a\in \mathbf{E}$.即$\exists n,f^{n}(a)=a$,


		则$f^{n+1}(a)=f(a),f^{n+2}(a)=f^{2}(a),…\Rightarrow\mid \mathbf{E}\mid=n$.这与$\mathbf{X}$是无限集合矛盾!证毕.
	\end{proof2}


	\begin{myproposition}{习题25}{}
		记区间$[0,1]$上的连续函数全体为$C[0,1]$,试证明:$\mid C[0,1]\mid =c$.	
	\end{myproposition}
	\begin{proof2}
		一方面,取常值函数集合$A=\{f\mid f(x)=a,a\in[1,2]\}$,则$A\subset C[0,1]$,故$\mid C[0,1]\mid \geq \mid A\mid =c.$
	

		另一方面,$\forall f\in C[0,1]$,记$A_{f}=\{f(x)\mid x\in [0,1]\bigcap \mathbf{Q}\}$.


		易见$A_{f}$中元素个数为$N_{0}$,是可列的.这样的$A_{f}$的个数为$c$.


		则$C[0,1]\to \displaystyle\bigcup_{f\in C[0,1]}A_{f}=\displaystyle\bigcup_{f\in C[0,1]}\{f(x_{1}),f(x_{2}),…\}$为单射.


		由第18题结论可知$\mid C[0,1]\mid \leq \mid \displaystyle\bigcup_{f\in C[0,1]}A_{f}\mid \leq c.$证毕.
	\end{proof2}


	\begin{myproposition}{习题26}{}
		记$\mathbf{R}$上一切实值函数的全体为$\psi $,试证明:$\mid \psi \mid=2^{c}$.	\end{myproposition}
	\begin{proof2}
		构造$f:2^{\mathbf{R}}\to \psi ,A\mapsto \mathbf{X_{A}}$.则$f$为单射,可得$\mid \psi \mid \leq 2^{c}$.


		构造$g:\psi\to 2^{\mathbf{R^{2}}}  ,f \mapsto \{x,f(x)\mid x \in \mathbf{R}\}$.则$g$为单射,可得$\mid \psi \mid \geq 2^{c}$.


		综上,$\mid \psi \mid=2^{c}$.证毕.


		$Tips:$25题是一条条平面内的曲线,26题是一条条空间内的曲线.25题取常值函数作为突破口,26题取特征函数作为突破口.25题用$f(x)$表征$x$,26题用$(x,f(x))$表征$f$.
		常值函数与特征函数在今后测度的学习中也常常作为问题的特殊情形,并能通过特征函数找到简单函数,解决可测函数的问题.
	\end{proof2}

    \section{$n$维欧氏空间}

    \begin{myproposition}{习题2}{}
        证明:$\mathbf{A} \subset \mathbf{R^{n}}$是开集$\Leftrightarrow \forall \mathbf{B}\subset \mathbf{R^{n}},\mathbf{A}\bigcap \overline{\mathbf{B}}\subset \overline{\mathbf{A}\bigcap \mathbf{B}}$
    \end{myproposition}
    \begin{proof2}
        注意到$\mathbf{A}\bigcap \overline{\mathbf{B}}=\mathbf{A}\bigcap(\mathbf{B}\bigcup\mathbf{B'})=(\mathbf{A}\bigcap\mathbf{B})\bigcup (\mathbf{A}\bigcap\mathbf{B'})$


        $\overline{\mathbf{A}\bigcap \mathbf{B}}=(\mathbf{A}\bigcap \mathbf{B})\bigcup (\mathbf{A}\bigcap \mathbf{B})'=(\mathbf{A}\bigcap \mathbf{B})\bigcup (\mathbf{A'}\bigcap \mathbf{B'})$


        故只需证明:$\mathbf{A} \subset \mathbf{R^{n}}$是开集$\Leftrightarrow \forall \mathbf{B}, \mathbf{A}\bigcap\mathbf{B'} \subset (\mathbf{A}\bigcap \mathbf{B})\bigcup (\mathbf{A'}\bigcap \mathbf{B'})$.


        充分性:当$\mathbf{A}$为开集时,$\mathbf{A} \subset \mathbf{A'}$,显然成立.


        必要性:取$\mathbf{B}=\mathbf{A^{c}}$,则$\mathbf{A}\bigcap \overline{\mathbf{A^{c}}}\subset \overline{\mathbf{A}\bigcap \mathbf{A^{c}}}=\emptyset$,故$\overline{\mathbf{A^{c}}}\subset \mathbf{A^{c}}$,又$\mathbf{A^{c}\subset\overline{\mathbf{A^{c}}} }$,则$\overline{\mathbf{A^{c}}}=\mathbf{A^{c}}$.


        所以$\mathbf{A^{c}}$为闭集,$\mathbf{A}$为开集. 

		可由此题引申出一些有意思的结论:


        若$\mathbf{G_{1}},\mathbf{G_{2}}$是$\mathbf{R^{n}}$中互不相交的开集,则$\mathbf{G_{1}}\bigcup \overline{\mathbf{G_{2}}}=\emptyset.$

        若$\mathbf{F_{1}},\mathbf{F_{2}}$是$\mathbf{R^{n}}$中互不相交的闭集,则$\exists \mathbf{G_{1}},\mathbf{G_{2}}$是$\mathbf{R^{n}}$中互不相交的开集满足
        $\mathbf{F_{1}\subset \mathbf{G_{1}}},\mathbf{F_{2}\subset \mathbf{G_{2}}}$.
    \end{proof2}


    \begin{myproposition}{习题4}{}
        若集合$\mathbf{A}\subset \mathbf{R^{n}}$只有孤立点,证明:$\mathbf{A}$是至多可数集.
    \end{myproposition}
    \begin{proof2}
        $\mathbf{A}\subset \mathbf{R^{n}}$只有孤立点$\Rightarrow \forall x \in \mathbf{A},\exists r>0,B_{r}(x)\bigcap\mathbf{A}={x}$.


        注意到$\mathbf{Q^{n}}$在$\mathbf{R^{n}}$中稠密.可取$y\in B_{r}(x)\bigcap\mathbf{Q^{n}}$,则$x\mapsto y,\mathbf{A} \to B_{r}(x)\bigcap\mathbf{Q^{n}}$为单射.


        故$\mid \mathbf{A}\mid \leq \mid\mathbf{Q^{n}} \mid =N_{0}$.证毕.



    \end{proof2}


    \begin{myproposition}{习题6}{}
        $\mathbf{A}\subset \mathbf{R^{n}}$可列,证明:$\exists x \in \mathbf{R^{n}},s.t.\mathbf{A}\bigcap (\mathbf{A}+x)=\emptyset$.
        其中$\mathbf{A}+x=\{x+y\mid y\in\mathbf{A}\}.$
    \end{myproposition}
    \begin{proof2}
        记$\mathbf{A}=\{r_{n}\}_{n=1}^{\infty}.$

        若$y \in \mathbf{A}\bigcap (\mathbf{A}+x)$,则$\exists m,n, s.t.r_{m}=r_{n}+x$.故$x=r_{m}-r_{n}$.

        全体$x$构成集合$\{r_{m}-r_{n}\}$.显然该集合为可数集.故$\exists x \in \mathbf{R^{n}},x \notin \{r_{m}-r_{n}\}$.证毕.
        
    \end{proof2}


    \begin{myproposition}{习题7,19}{}
        设集合$\mathbf{A}\subset \mathbf{R^{n}}$.证明:从$\mathbf{A}$的任意一个开覆盖中可以取出可列个子覆盖.
        
        
        推论:设集合$\mathbf{A}\subset \mathbf{R^{n}}$.若$\forall x\in \mathbf{R^{n}},\exists r,s.t.\mathbf{A}\bigcap \mathbf{B_{r}(x)}$为可列集,则$\mathbf{A}$是可列集.
    \end{myproposition}
    \begin{proof2}
        有限覆盖定理告诉我们,有界闭集的开覆盖中可以取出有限个子覆盖.而命题1.3.4则是将此结论推广为“任意集合的开覆盖中可以取出可列个子覆盖”.$\{B(x,\sigma _{x})\mid x\in \mathbf{A}\}$是$\mathbf{A}$的一个开覆盖.
        从每个$\{B(x,\sigma _{x})\}$中取出一个有理数点代表这个球邻域,可知任意集合的开覆盖中可以取出可列个子覆盖.证毕.

        利用习题19的结论,习题7是显然的.这里不再赘述.
    \end{proof2}

    \begin{myproposition}{习题8}{}
        设集合$\mathbf{A},\mathbf{B} \subset \mathbf{R}$.证明:$$(\mathbf{A}\times \mathbf{B})'=(\overline{\mathbf{A}}\times\mathbf{B'})\bigcup (\mathbf{A'}\times\overline{\mathbf{B}}).$$
    \end{myproposition}
    \begin{proof2}
        $(a,b)\in(\mathbf{A}\times \mathbf{B})'$ 
        
        $\Leftrightarrow(a,b)\in \overline{(\mathbf{A}\times \mathbf{B})\backslash{\{(a,b)\}}}$

        $\Leftrightarrow (a,b)\in \overline{(\mathbf{A}\backslash{\{a\}})\times (\mathbf{B}\backslash{\{b\}})}     $
        
        $\Leftrightarrow (a,b)\in \overline{(\mathbf{A}\backslash{\{a\}})\times \mathbf{B}} \bigcup \overline{ \mathbf{A}\times (\mathbf{B}\backslash{\{b\}})}$
        
        $\Leftrightarrow \exists \{(a_{n},b_{n})\}\subset (\mathbf{A}\backslash{\{a\}})\times \mathbf{B},s.t.(a_{n},b_{n}) \to (a,b)$或$\exists \{(a_{n},b_{n})\}\subset  \mathbf{A}\times (\mathbf{B}\backslash{\{b\}}),s.t.(a_{n},b_{n}) \to (a,b)$
   
        $\Leftrightarrow \exists \{a_{n}\}\subset \mathbf{A}\backslash\{a\},\{b_{n}\}\subset\mathbf{B},s.t.\{a_{n}\}\to a,\{b_{n}\} \to b$或$\exists \{a_{n}\}\subset \mathbf{A},\{b_{n}\}\subset\mathbf{B}\backslash\{b\},s.t.\{a_{n}\}\to a,\{b_{n}\} \to b.$

        $\Leftrightarrow a\in\mathbf{A'},b \in \overline{\mathbf{B}}$或$a\in \overline{\mathbf{A}},b\in \mathbf{B'}$

        $\Leftrightarrow (a,b) \in (\overline{\mathbf{A}}\times\mathbf{B'})\bigcup (\mathbf{A'}\times\overline{\mathbf{B}})$.证毕.

    \end{proof2}

    \begin{myproposition}{习题10}{}
		设$f\in C^{1}[a,b]$,令$$\mathbf{E}=\{x\in[a,b]\mid f(x)=0\}\bigcup\{x\in[a,b]\mid f'(x)>0\}.$$
		证明:$\mathbf{E}$中每一点皆是$\mathbf{E}$中的孤立点.
	\end{myproposition}
    \begin{proof2}
		考虑反证法.假设$x\in\mathbf{E}$不是$\mathbf{E}$的孤立点,则$x\in\mathbf{E'}$.	
        
        则$\exists \{x_{n}\} \subset \mathbf{E}\backslash{x},s.t.x_{n}\to x.$

		此时$f'(x)=\displaystyle\lim_{n \to \infty}\displaystyle\frac{f(x_{n})-f(x)}{x_{n}-x}=0,$与$x\in \mathbf{E}$矛盾.
		
		故而$\mathbf{E}$中每一点皆是$\mathbf{E}$中的孤立点.
	\end{proof2}

	\begin{myproposition}{习题13}{}
		证明:$\mathbf{R}$上任何实函数$f$的连续点之集是$G_{\delta }$集.
	\end{myproposition}
	\begin{proof2}
		先证明几个引理.

		\textsl{引理1:设$f(x)$是$R$上的实函数,则对于$\forall \varepsilon>0,\{x\mid w(x)<\varepsilon\}$为开集.其中$w(x)=\displaystyle\lim_{\delta\to 0^{+}} sup \{\mid f(x_{1})-f(x_{2})\mid \mid x_{1},x_{2}\in O(x,\delta)\}.$ }
	
		引理1的证明:记$E=\{x\in\mathbf{R}\mid w(x)<\varepsilon\}.$任取$x_{0}\in\mathbf{E},$则$w(x_{0})<\varepsilon$.由极限的保号性知,$\exists \delta_{0}>0,s.t.sup\{\mid f(x_{1})-f(x_{2})\mid x_{1},x_{2}\in O(x_{0},\delta_{0})\}$.

		易见$O(x_{0},\displaystyle\frac{1}{2}\delta_{0})\subset \mathbf{E}$.故$x_{0}$是$\mathbf{E}$的内点,即$\mathbf{E}$是开集.

		\textsl{引理2:设$f(x)$是$R$上的实函数,$f(x)$在$x_{0}$处连续的充要条件为$w(x_{0})=0$.}

		引理2的证明:
		
		必要性:$f(x)$在$x_{0}$处连续$\Rightarrow,\forall \varepsilon>0,\exists \delta_{0}>0$,当$x\in O(x_{0},\delta_{0})$时,$\mid f(x)-f(x_{0})\mid<\displaystyle\frac{\varepsilon}{4}.$

		故当 $x_{1},x_{2}\in O(x,\delta),\mid f(x_{1})-f(x_{2})\mid \leq \mid f(x_{1})-f(x_{0})\mid + \mid f(x_{2})-f(x_{0})\mid < \displaystyle\frac{\varepsilon}{2}.$
		所以当$0<\delta\leq \delta_{0}$时,$sup\{\mid f(x_{1})-f(x_{2})\mid x_{1},x_{2}\in O(x_{0},\delta)\}\leq \varepsilon.$

		故$w(x)=\displaystyle\lim_{\delta\to 0^{+}}sup\{\mid f(x_{1})-f(x_{2})\mid \mid x_{1},x_{2}\in O(x,\delta)\}=0.$ 
	
		充分性:$w(x)=0 \Rightarrow \forall \varepsilon,\exists \delta_{0}>0,s.t.$当$0<\delta\leq\delta_{0}$时,$sup\{\mid f(x_{1})-f(x_{2})\mid x_{1},x_{2}\in O(x_{0},\delta)\}< \varepsilon.$
		取$x_{1}=x,x_{2}=x_{0}$,得$\mid f(x)-f(x_{0})\mid<\varepsilon \Rightarrow f(x)$在$x_{0}$处连续.
		
		再证明此题.

		由引理1,引理2可知:$\mathbf{E}=\displaystyle\bigcap_{n=1}^{\infty}\{x\in\mathbf{R}\mid w(x)<\displaystyle\frac{1}{n}\}$,故$\mathbf{E}$是$G_{\delta }$集.

	\end{proof2}


	\begin{myproposition}{习题15}{}
		设$\{\mathbf{G_{k}}\}$是$\mathbf{R^{n}}$中开集的升列,有界闭集$\mathbf{F}$是$\displaystyle\bigcup_{k}\mathbf{G_{k}}$的子集,证明:$\mathbf{F}$含于某个$\mathbf{G_{k}}$中.
	\end{myproposition}
	\begin{proof2}
		\textsl{容易认为$\mathbf{F}$包含于一个比较大的$\mathbf{G_{k}}$中.有多大?如何刻画多大?}
       
		$\displaystyle\bigcup_{k}\mathbf{G_{k}}$是$\mathbf{F}$的开覆盖,由有限覆盖定理,可以从中取出$\{\mathbf{G_{K_{1}}},\mathbf{G_{k_{2}},…,\mathbf{G_{k_{n}}}}\}$使得
		$\mathbf{F}\subset \displaystyle\bigcup_{t=1}^{n}\mathbf{G_{kt}}$.

		由$\mathbf{G_{K_{1}}}\subset\mathbf{G_{k_{2}}\subset…\subset\mathbf{G_{k_{t}}}}$知$\mathbf{F}\subset \displaystyle\bigcup_{t=1}^{n}\mathbf{G_{kt}}=\mathbf{G_{kn}}$.证毕.

	\end{proof2}

	\begin{myproposition}{习题16}{}
		设$\mathbf{F_{k}}$是$\mathbf{R^{n}}$中有界闭集的下降列.若$\mathbf{G}$是一个开集满足$\displaystyle\bigcap_{k}\mathbf{F_{k}}\subset \mathbf{G}$,证明:$\mathbf{G}$包含某个$\mathbf{F_{k}}.$
	\end{myproposition}
    \begin{proof2}
		\textsl{类比上一题,发现此题证明需要条件$\mathbf{G^{c}}$为有界闭集.如何找到这样一个和$\mathbf{G^{c}}$性质相似的有界闭集?}

		考虑取辅助集——有界闭集$\mathbf{E},s.t.\mathbf{G}\subset \mathbf{F_{1}}\subset\mathbf{E}$.可以看出$\mathbf{E}\bigcap\mathbf{G^{c}}$为有界闭集.
		利用补集性质转化已有条件,可知:$\mathbf{E^{c}}\bigcup \mathbf{G}$包含某个$\mathbf{F_{k}}$.

		结合待证,我们希望证明更强的:$\mathbf{E^{c}}$不包含$\mathbf{F_{k}}$.这由我们的构造是显然的.

		易见$\{\mathbf{F_{k}}\}$为开集升列,$\mathbf{E}\bigcap\mathbf{G^{c}}$为有界闭集,$\displaystyle\bigcap_{k}\mathbf{F_{k}} \subset \mathbf{G} \subset \mathbf{G}\bigcup \mathbf{E^{c}}.$
		$\Rightarrow \mathbf{E}\bigcap \mathbf{G^{c}} \subset \displaystyle\bigcup_{k}\mathbf{F_{k}^{c}}$,由上题结论,不难有$\mathbf{F_{k_{n}}}\subset \mathbf{E^{c}}\bigcup \mathbf{G}.$
		而$\mathbf{E^{c}}$不包含$\mathbf{F_{k}}$,故$\mathbf{F_{k_{n}}} \subset \mathbf{G}$.证毕.
	\end{proof2}
	
	\begin{myproposition}{习题18}{}
		设$\mathbf{F_{\alpha}}$是$\mathbf{R^{n}}$中一族有界闭集,若任取其中有限个$\mathbf{F_{\alpha_{1}}}$、$\mathbf{F_{\alpha_{2}}}$、……、$\mathbf{F_{\alpha_{n}}}$,都有$\displaystyle \bigcap_{i=1}^{m} {\mathbf{F_{\alpha_{i}}}}\neq \emptyset.$
		证明:$\bigcap{\mathbf{F_{\alpha}}}\neq \emptyset$.
	\end{myproposition}
    \begin{proof2}
		由题中关键词“有界闭集”、“有限”容易联想到有限覆盖定理.

		反证法.假设$\bigcap{\mathbf{F_{\alpha}}} = \emptyset$.则$\bigcup \mathbf{F_{\alpha}}=\mathbf{R^{n}}$.故对于某个$\mathbf{F_{\alpha _{0}}}$,$\bigcup F_{\alpha}$是其开覆盖.

		由有限覆盖定理,知$\exists \mathbf{F_{\alpha_{i}}^{c}},\displaystyle\mathbf{F_{\alpha _{0}}} \subset \bigcup_{i=1}^{n}F_{\alpha_{i}}^{c}$, 
		即$\displaystyle\bigcap_{i=1}^{n}\mathbf{F_{\alpha_{i}}}\subset \mathbf{F_{\alpha_{0}}^{c}}.$故$\mathbf{F_{\alpha _{0}}} \cap  \displaystyle \bigcap_{i=1}^{n}\mathbf{F_{\alpha_{i}}}=\emptyset$.这与题设矛盾!

		证毕!
	\end{proof2}


	\begin{myproposition}{习题26}{}
		$\mathbf{E}$是$\mathbf{F}$的任一无限子集,证明:$\mathbf{F}\subset\mathbf{R_{n}}$是有界闭集当且仅当$\mathbf{E'}\bigcap\mathbf{F}\neq \emptyset$.
	\end{myproposition}
    \begin{proof2}
        1° 必要性

		反证.假设$\mathbf{E'}\bigcap\mathbf{F}=\emptyset$.由$\mathbf{F}$为闭集且$\mathbf{E}$为$\mathbf{F}$的子集
		可知$\mathbf{E'}\subset\mathbf{F'}\subset\mathbf{F}$,故$\mathbf{E'}\bigcap\mathbf{F}=\mathbf{E'}$.即$\mathbf{E'}=\emptyset$.

		由$Bolzano\quad Weierstrass$极限点定理可知,有界无限集合必有聚点,即$\mathbf{E'} \neq 0$.矛盾!故$\mathbf{E'}\bigcap\mathbf{F}\neq \emptyset$.

		2°充分性

		反证.若$\mathbf{F}$无界(直观上感觉可以取出足够分散的子集$\mathbf{E}$,使得$\mathbf{E'}=\emptyset$).
		
		取$\mathbf{E}=\{x_{k}\mid k\geq 1\} \subset \mathbf{F_{1}}$,使得$d(x_{k},0)>k$且$d(x_{k},x_{p})>1,\forall p\neq k$.

		易见此时$\mathbf{E'}=\emptyset$,则$\mathbf{E'}\bigcap\mathbf{F}=\emptyset$,矛盾!故$\mathbf{F}$有界.

		若$\mathbf{F}$中存在某收敛点列$\{x_{k}\},x_{k}\to x$,取$\mathbf{E}=\{x_{k}\}$,则$\mathbf{E'}=\{x\}$.

		又$\mathbf{E'}\bigcap\mathbf{F}\neq \emptyset$,故$x\in\mathbf{F}$.
	    这表明$\mathbf{F}$中所有收敛点列所收敛到的聚点都在$\mathbf{F}$中,$\mathbf{F}$为闭集.

		证毕!
	\end{proof2}



















































\end{document}